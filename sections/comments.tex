\domenico{
\begin{itemize}
\item It is unclear why you need one layer for FFT magnitude and two layers for I/Q samples. What's the intuition? Can you use more layers also for FFT magnitude? Some discussion on the number of layers is done later, by it is not referenced while you introduce Fig1 and Fig2. 
\sreeraj{Added classification accuracy details for varying layer-depths and cell numbers for mag-FFT model too.}
\sofie{But this it is still confusing that you introduce a 1-layer for mag-FFT and 2-layer for IQ...}

\item A short intro to Sec III is also necessary. 
\sreeraj{Added a single line introduction.}
\item You present Fig. 3 and say that you cannot differentiate between LTE and DVB-T. As you said, this is expected, but I would also mention that this is expected because both are based on OFDM modulations. Right?
\sreeraj{Yes. Updated the argument in that section.}

\item In addition to that, it seems that you just give up and don't find a why to differentiate these technologies. One option is to switch and request IQ samples, the second could be to request smaller averaging time. But, in summary, I think you should propose a solution to differentiate the above technologies.
\sreeraj{Added switching to IQ pipeline as a solution. Averaging cannot help in the case of technologies with same PSD.}

\item Can you somehow unify Fig.1 and Fig.2 ? There are of components which are the same, and based on the input data, the number of layers is selected. It would give a more unified approach to the solution you propose.
\sreeraj{I can do that, but I it might complicate the input stream representation. Suggestions?}
\sofie{I agree you should combine. I also think you need a good notation for at or af, and relate it to the r(t) in the Eqs. 2 and 3. Can you come up with a generic model, where you enable both time and frequency domain representation (can you compare both) and averaging, and amplitude versus phase. For Mag-FFT, you throw away phase and also average, for compression reasons. }

\item The best would be of course to have a new section where all the components are put together in Electrosense, with the parameters such as number of layers and cell size that provide the right trade-off in performance in Electrosense network using ternary weights for IQ samples (when these are needed), and tested again. Most likely there will be new insights from the experimental evaluation (delays to make a decisions, reconfigure the nodes to switch to IQ samples, total amount of bandwidth requested before having a high confidence in your results, etc). This is probably quite some work.
\sreeraj{Indeed it is quite some work. We need to develop optimized implementations and the look into the processing and memory trade-offs. This is close to Vincent's second last comment. I will try to do an initial analysis at a theoretical level.}
\item In Conclusion section, I wouldn't say that frequency hopping and DSSS are more difficult. I would say that a methodology that can handle all possible spread spectrum modulations should be used.
\sreeraj{Done}
\end{itemize}
}

\vincent{
\begin{itemize}
\item You mention in the intro that spectrum monitoring is "extremely" challenging and that it is "multidisciplinary". You should give some arguments to back up these claims as it is not clear given the paper content.
\sreeraj{Done}
\item In the Intro, you say: we designed Electrosense. I would not write it like this as not all people that designed Electrosense are authors of this paper. You can rather say: "Electrosense was designed to address these challenges..."
\sreeraj{Done}
\item In Section II.A, you should better describe the Electrosense dataset. For example with information such as the sampling rate, dynamic range, averaging factors, sampling bandwidth, sweeping stratefy etc. Please also mention if the sensors were indoor/outdoor and that the captured signals were real signals captured over the air with an omnidirectional/discone? antenna.
\sreeraj{Done}
\item In section II.A: 60 seconds is not the time averaging but the time resolution. You should mention the averaging as well (I think 5 or 10 by default?). It is further not clear how you combine the averaged magnitude FFT from different bands as the sensor is constantly sweeping. I guess you combined different FFT scans since the technologies you analyze have a larger bandwidth than the sampling bandwidth of 2.4 MHz.
\sreeraj{Yes. Added more details on the dataset section to explain the combining too. Done.}
\item In Section II.B please explain how the RadioML2016 signals have been generated and captured (through a real radio frontend? If yes, please describe this acquisition setup.
\sreeraj{Added. Done.}
\item You never explicitly mention the classification problems you are trying to solve. I would introduce a new Section just before Section III to formalize the technology classification problem with averaged magnitude FFT data and the signal modulation classification problem with raw IQ data.
\sreeraj{Added a new section with problem formulation.}
\item Section III.A: It is not clear why you use only a single layer model for technology classification and higher layer models for complex signals:
\sreeraj{Included classification results for varying mag-FFT model layer depth and cell numbers. Added more details in the text.}
\item In Section III.B: You should mention here that you investigate 1, 2, and 3 layer models in this work (and not only 2 layers as stated here)
\sreeraj{Added. Done.}
\item In Section III.B: You should mention the option of using IQ samples instead of amplitude and phase as input and why IQ samples is expected to be worse. If you have the classification performance results with IQ samples, I would include them as well in the evaluation to show the impact of this design choice.
\sreeraj{Our standard 2 layer lstm model doesn't even converge during training when fed  with IQ data, giving a constant accuracy close to 9\% for 11 modulation schemes. Ran the models again on IQ data without any success. Updated this 9\% figure in the text. Not worth adding to the plots.}
\item In Figure 3: The dvb classification performance does not add to 1? Is this possible that the outcome is "not classified". All other classifications add up to 1!
\sreeraj{Nice catch. There was a bug in the confusion matrix plotting code for mag-FFT case. Fixed.}
\item In Section IV, first paragraph, you mention that in Electrosense the spectrum is averaged for a period of 60 seconds. This is not entirely true since the sensor is inspecting a specific band only every ~50 seconds. At time resolution of 60 seconds, the spectrum should be averaged over only two sensor scans and not the entire 60 seconds.
\sreeraj{True. Corrected it.}
\item In Section IV, second paragraph, Please explain in more details why the classification with magnitude FFT does not work for modulation classification.
\sreeraj{Added some more details. I can also add one figure showing same PSDs for multiple modulation schemes if it is required. Suggestions?}
\item I would put subsection IV.A in section III.
\sreeraj{Moved.}
\item Section IV.B, last sentence: Explain how you "normalize" amplitude and phase.
\sreeraj{Added normalization details.}
\item Figure 7: I would put here the performance of the 3-layer model since you have the results.
\sreeraj{Added.}
\item Section IV.F: I am missing here a result on computation efficiency. Can you provide results on CPU load of the different models? Currently, you do not really show that the quantized model is less resource intensive and that it works on a RPi platform.
\sreeraj{Evaluated binarized CNN model and gave some more insights how the performance improvements can be achieved on general/special purpose hardware. Proper resource reduction requires proper optimized implementation on RPi and benchmarking which can easily consume close to three weeks.}
\item The title is a little bit complex and not very self-explanatory. I will think of another candidate
\sreeraj{suggestions?}
\end{itemize}
}

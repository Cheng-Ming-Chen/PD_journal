%Adapted to 200 words
This paper looks into the modulation classification problem for a distributed wireless spectrum sensing network. First, a new data-driven model for \ac{amc} based on long short term memory (LSTM) is proposed. The model learns from the time domain amplitude and phase information of the modulation schemes present in the training data without requiring expert features like higher order cyclic moments. Analyses show that the proposed model yields an average classification accuracy of close to 90\% at varying SNR conditions ranging from 0dB to 20dB. %, independent of the channel characteristics. 
Further, we explore the utility of this LSTM model for a variable symbol rate scenario. We show that a LSTM based %variable length 
model can learn good representations of variable length time domain sequences, which is useful in classifying modulation signals with different symbol rates. The achieved accuracy of 75\% on an input sample length of 64 for which it was not trained, substantiates the representation power of the model. To reduce the data communication overhead from distributed sensors, the feasibility of classification using averaged magnitude spectrum data and on-line classification on the low-cost spectrum sensors are studied. %Furthermore, performance of quantized realizations of the proposed models are analyzed to reduce the processing power requirements enabling its deployment on sensors with low processing power.
Furthermore, quantized realizations of the proposed models are analyzed for deployment on sensors with low processing power.
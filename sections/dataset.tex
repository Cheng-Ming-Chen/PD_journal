\begin{table}[!t]
\begin{center}
\begin{tabular}{|l|l|}
	\hline
    Modulations     & 8PSK, AM-DSB, AM-SSB, BPSK,\\
    								  & CPFSK, GFSK, PAM4, QAM16,\\
    								  & QAM64, QPSK, WBFM\\
    \hline
  	Samples per symbol &   4 \\
  	\hline	
    Sample length &   128 \\
  	\hline
    SNR Range &  -20dB to +20dB \\
  	\hline
   	Number of training samples &   82500 vectors\\
  	\hline
  	Number of test samples &  82500 vectors\\
    \hline
\end{tabular}
\end{center}
\caption{RadioML2016.10a dataset parameters.}
\label{table_rml_dataset}
\end{table}

\begin{table}[!t]
\begin{center}
\begin{tabular}{|l|l|}
	\hline
    Modulations     & 8PSK, AM-DSB, AM-SSB, BPSK,\\
    								  & CPFSK, GFSK, PAM4, QAM16,\\
    								  & QAM64, QPSK, WBFM\\
    \hline
  	Samples per symbol &   4 and 8 sps\\
  	\hline	
    Sample length &   128 to 512\\
  	\hline
    SNR Range &  -20dB to +20dB \\
  	\hline
  	Number of training samples &   165000 vectors\\
  	\hline
  	Number of test samples &  165000 vectors\\
    \hline
\end{tabular}
\end{center}
\caption{Modified complex RadioML dataset parameters.}
\label{table_modrml_dataset}
\end{table}

A publicly available dataset used for evaluating the performance of the proposed model is detailed in this section. The standard dataset is also extended to evaluate the sample rate dependence of the proposed model.

\subsection{RadioML dataset}
A standard modulation dataset presented in \cite{o2016radio} is used as the baseline for training and evaluating the performance of the proposed classifier. The used RadioML2016.10a dataset is a synthetically generated dataset using GNU Radio \cite{gnuradio_web} with commercially used modulation parameters. This dataset also includes a number of realistic channel imperfections such as channel frequency offset, sample rate offset, additive white gaussian noise along with multipath fading. It contains modulated signals with 4~\ac{sps} and a sample length of 128 samples. Used modulations along with the complete parameter list can be found in Table~\ref{table_rml_dataset}. Detailed specifications and generation details of the dataset can be found in \cite{o2016radio}.

\subsection{Modified RadioML dataset}
The standard radioML dataset is extended using the generation code\footnote{https://github.com/radioML/dataset} by varying the \textit{samples per symbol} and \textit{sample length} parameters for evaluating the sample rate dependencies of the \ac{lstm} model. The extended parameters of the used dataset are listed in the Table~\ref{table_modrml_dataset}. The extended dataset contains signals with 4 and 8 samples per symbol. This dataset is generated to evaluate the robustness of the model in varying symbol rate scenarios. 



